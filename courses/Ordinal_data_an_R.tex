% Options for packages loaded elsewhere
\PassOptionsToPackage{unicode}{hyperref}
\PassOptionsToPackage{hyphens}{url}
\PassOptionsToPackage{dvipsnames,svgnames,x11names}{xcolor}
%
\documentclass[
  letterpaper,
  DIV=11,
  numbers=noendperiod]{scrartcl}

\usepackage{amsmath,amssymb}
\usepackage{iftex}
\ifPDFTeX
  \usepackage[T1]{fontenc}
  \usepackage[utf8]{inputenc}
  \usepackage{textcomp} % provide euro and other symbols
\else % if luatex or xetex
  \usepackage{unicode-math}
  \defaultfontfeatures{Scale=MatchLowercase}
  \defaultfontfeatures[\rmfamily]{Ligatures=TeX,Scale=1}
\fi
\usepackage{lmodern}
\ifPDFTeX\else  
    % xetex/luatex font selection
\fi
% Use upquote if available, for straight quotes in verbatim environments
\IfFileExists{upquote.sty}{\usepackage{upquote}}{}
\IfFileExists{microtype.sty}{% use microtype if available
  \usepackage[]{microtype}
  \UseMicrotypeSet[protrusion]{basicmath} % disable protrusion for tt fonts
}{}
\makeatletter
\@ifundefined{KOMAClassName}{% if non-KOMA class
  \IfFileExists{parskip.sty}{%
    \usepackage{parskip}
  }{% else
    \setlength{\parindent}{0pt}
    \setlength{\parskip}{6pt plus 2pt minus 1pt}}
}{% if KOMA class
  \KOMAoptions{parskip=half}}
\makeatother
\usepackage{xcolor}
\setlength{\emergencystretch}{3em} % prevent overfull lines
\setcounter{secnumdepth}{5}
% Make \paragraph and \subparagraph free-standing
\ifx\paragraph\undefined\else
  \let\oldparagraph\paragraph
  \renewcommand{\paragraph}[1]{\oldparagraph{#1}\mbox{}}
\fi
\ifx\subparagraph\undefined\else
  \let\oldsubparagraph\subparagraph
  \renewcommand{\subparagraph}[1]{\oldsubparagraph{#1}\mbox{}}
\fi


\providecommand{\tightlist}{%
  \setlength{\itemsep}{0pt}\setlength{\parskip}{0pt}}\usepackage{longtable,booktabs,array}
\usepackage{calc} % for calculating minipage widths
% Correct order of tables after \paragraph or \subparagraph
\usepackage{etoolbox}
\makeatletter
\patchcmd\longtable{\par}{\if@noskipsec\mbox{}\fi\par}{}{}
\makeatother
% Allow footnotes in longtable head/foot
\IfFileExists{footnotehyper.sty}{\usepackage{footnotehyper}}{\usepackage{footnote}}
\makesavenoteenv{longtable}
\usepackage{graphicx}
\makeatletter
\def\maxwidth{\ifdim\Gin@nat@width>\linewidth\linewidth\else\Gin@nat@width\fi}
\def\maxheight{\ifdim\Gin@nat@height>\textheight\textheight\else\Gin@nat@height\fi}
\makeatother
% Scale images if necessary, so that they will not overflow the page
% margins by default, and it is still possible to overwrite the defaults
% using explicit options in \includegraphics[width, height, ...]{}
\setkeys{Gin}{width=\maxwidth,height=\maxheight,keepaspectratio}
% Set default figure placement to htbp
\makeatletter
\def\fps@figure{htbp}
\makeatother

\KOMAoption{captions}{tableheading}
\makeatletter
\@ifpackageloaded{tcolorbox}{}{\usepackage[skins,breakable]{tcolorbox}}
\@ifpackageloaded{fontawesome5}{}{\usepackage{fontawesome5}}
\definecolor{quarto-callout-color}{HTML}{909090}
\definecolor{quarto-callout-note-color}{HTML}{0758E5}
\definecolor{quarto-callout-important-color}{HTML}{CC1914}
\definecolor{quarto-callout-warning-color}{HTML}{EB9113}
\definecolor{quarto-callout-tip-color}{HTML}{00A047}
\definecolor{quarto-callout-caution-color}{HTML}{FC5300}
\definecolor{quarto-callout-color-frame}{HTML}{acacac}
\definecolor{quarto-callout-note-color-frame}{HTML}{4582ec}
\definecolor{quarto-callout-important-color-frame}{HTML}{d9534f}
\definecolor{quarto-callout-warning-color-frame}{HTML}{f0ad4e}
\definecolor{quarto-callout-tip-color-frame}{HTML}{02b875}
\definecolor{quarto-callout-caution-color-frame}{HTML}{fd7e14}
\makeatother
\makeatletter
\makeatother
\makeatletter
\makeatother
\makeatletter
\@ifpackageloaded{caption}{}{\usepackage{caption}}
\AtBeginDocument{%
\ifdefined\contentsname
  \renewcommand*\contentsname{Table of contents}
\else
  \newcommand\contentsname{Table of contents}
\fi
\ifdefined\listfigurename
  \renewcommand*\listfigurename{List of Figures}
\else
  \newcommand\listfigurename{List of Figures}
\fi
\ifdefined\listtablename
  \renewcommand*\listtablename{List of Tables}
\else
  \newcommand\listtablename{List of Tables}
\fi
\ifdefined\figurename
  \renewcommand*\figurename{Figure}
\else
  \newcommand\figurename{Figure}
\fi
\ifdefined\tablename
  \renewcommand*\tablename{Table}
\else
  \newcommand\tablename{Table}
\fi
}
\@ifpackageloaded{float}{}{\usepackage{float}}
\floatstyle{ruled}
\@ifundefined{c@chapter}{\newfloat{codelisting}{h}{lop}}{\newfloat{codelisting}{h}{lop}[chapter]}
\floatname{codelisting}{Listing}
\newcommand*\listoflistings{\listof{codelisting}{List of Listings}}
\makeatother
\makeatletter
\@ifpackageloaded{caption}{}{\usepackage{caption}}
\@ifpackageloaded{subcaption}{}{\usepackage{subcaption}}
\makeatother
\makeatletter
\@ifpackageloaded{tcolorbox}{}{\usepackage[skins,breakable]{tcolorbox}}
\makeatother
\makeatletter
\@ifundefined{shadecolor}{\definecolor{shadecolor}{rgb}{.97, .97, .97}}
\makeatother
\makeatletter
\makeatother
\makeatletter
\makeatother
\ifLuaTeX
  \usepackage{selnolig}  % disable illegal ligatures
\fi
\IfFileExists{bookmark.sty}{\usepackage{bookmark}}{\usepackage{hyperref}}
\IfFileExists{xurl.sty}{\usepackage{xurl}}{} % add URL line breaks if available
\urlstyle{same} % disable monospaced font for URLs
\hypersetup{
  pdftitle={Ordinal Data Analysis in R},
  pdfauthor={Matteo Ventura},
  colorlinks=true,
  linkcolor={blue},
  filecolor={Maroon},
  citecolor={Blue},
  urlcolor={Blue},
  pdfcreator={LaTeX via pandoc}}

\title{Ordinal Data Analysis in R}
\usepackage{etoolbox}
\makeatletter
\providecommand{\subtitle}[1]{% add subtitle to \maketitle
  \apptocmd{\@title}{\par {\large #1 \par}}{}{}
}
\makeatother
\subtitle{Measuring Human Perceptions from Surveys}
\author{Matteo Ventura}
\date{2025-04-16}

\begin{document}
\maketitle
\ifdefined\Shaded\renewenvironment{Shaded}{\begin{tcolorbox}[boxrule=0pt, interior hidden, sharp corners, borderline west={3pt}{0pt}{shadecolor}, frame hidden, enhanced, breakable]}{\end{tcolorbox}}\fi

Description of the course

Surveys are key tools for measuring human perceptions, capturing latent
traits through structured responses. Among the data they generate,
ordinal and rating data are particularly important yet often less
studied, requiring specialized statistical techniques. Ordinal data
appears frequently in real-world applications, such as customer
satisfaction surveys, psychological assessments, and medical research,
making its correct analysis crucial for obtaining reliable insights.
This short course provides instructor-led, hands-on training in the
analysis of ordinal data. It begins with an overview of survey design
and the validation of results, focusing on building effective surveys
and ensuring the reliability of the data obtained. The course then
covers the most commonly used statistical models for analyzing ordinal
data, with an emphasis on discovering latent patterns and traits. Both
theoretical foundations and practical applications will be explored,
using real-world case studies from domains such as marketing, social
sciences, tourism and culture.

A common approach to analyzing ordinal data is to treat it as numerical,
but this can lead to a loss of statistical power. In this course,
participants will learn how to apply specialized methods designed for
ordinal data, allowing them to draw more effective and reliable
conclusions.

Objectives of the course

By the end of the course, participants will have both theoretical
knowledge and practical skills to analyze ordinal data in research and
professional settings. Specifically, they will be able to:

\begin{itemize}
\tightlist
\item
  Understand what ordinal data is, how it differs from other types of
  data, and the challenges involved in its analysis
\item
  Compute and interpret reliability and validity measures
\item
  Fit proportional odds models in R and interpret the results
\item
  Analyse rating data by applying CUB models
\end{itemize}

\hypertarget{introduction-to-ordinal-data-and-survey-design}{%
\section{Introduction to Ordinal Data and Survey
Design}\label{introduction-to-ordinal-data-and-survey-design}}

\hypertarget{the-role-of-measurement-in-science}{%
\subsection{The Role of Measurement in
Science}\label{the-role-of-measurement-in-science}}

Measurement is a fundamental activity in science, indeed we acquire
knowledge about the world around us by observing it, and we usually
quantify to give a sense to what we observe. Therefore, measurement is
essential in a wide range of research contexts.

There exist several situations in which scientists come up with
measurement problems, even though they are not interested primary in
measurement. For instance:

\begin{enumerate}
\def\labelenumi{\arabic{enumi})}
\item
  A health psychologist needs a measurement scale which doesn't seem to
  exist. The study depends on a tool that can clearly distinguish
  between what individuals want to happen and what they expect to happen
  when visiting a physician. However, the review of previous research
  reveals that existing scales often blur this distinction,
  unintentionally mixing the two concepts. None of the available
  instruments capture the separation in the specific way her study
  requires. While the psychologist could create a few items that appear
  to address the difference between wants and expectations, she/he is
  concerned that these improvised questions may lack the reliability and
  validity necessary to serve as accurate measures.
\item
  An epidemiologist is conducting secondary analyses on data from a
  national health survey. They wish to investigate the link between
  perceived psychological stress and health status. Unfortunately, the
  survey did not include a validated stress measure. While it may be
  possible to construct one using existing items, a poorly constructed
  scale could lead to misleading conclusions.
\item
  A marketing team is struggling to design a campaign for a new line of
  high-end infant toys. Focus groups suggest that parents are heavily
  influenced by a toy's perceived educational value. The team
  hypothesizes that parents with strong educational and career
  aspirations for their children are more likely to be interested in the
  product. To test this idea across a broad, geographically diverse
  sample, the team needs a way to reliably measure parental aspirations.
  Something that additional focus groups can't easily provide.
\end{enumerate}

Despite coming from different disciplines, these researchers share a
common understanding: using arbitrary or poorly designed measurement
tools increases the risk of collecting inaccurate data. As a result,
developing their own carefully constructed measurement instruments
appears to be the most reliable solution.

Historically, measurement problems were well-known in natural sciences
such as physics and astronomy, even concerning figures like Isaac
Newton. However, among social scientists, a debate arose regarding the
measurability of psychological variables. While physical attributes like
mass and length seem to possess an intrinsic mathematical structure
similar to positive real numbers, the measurement of psychological
variables was considered impossible by the Commission of the British
Association for the Advancement of Science. The primary reason was the
difficulty in objectively ordering or summing sensory perceptions, as
well illustrated by the question: how can one establish that a sensation
of ``a little warm'' plus another similar sensation equals ``twice as
warm''?

\hypertarget{measurement-classification}{%
\subsubsection*{Measurement
classification}\label{measurement-classification}}
\addcontentsline{toc}{subsubsection}{Measurement classification}

The americal psychologist Stevens (1946) disagreed with this
perspective. He contended that the rigid requirement of ``strict
additivity,'' as seen in measurements of length or mass, was not
essential for measuring sensations. He pointed out that individuals
could make reasonably consistent ratio judgments regarding the loudness
of sounds. For instance, they could determine if one sound was twice as
loud as another.

Stevens further argued that this ``ratio'' characteristic enabled the
data derived from such measurements to be mathematically analyzed. He is
known for categorizing measurements into nominal, ordinal, interval, and
ratio scales. In his view, judgments about sound ``loudness'' belonged
to the ratio scale.

Despite the classification proposed by Stevens has been criticized by
several authors and new classifications has been proposed, it is the
most commonly accepted and used internationally.

Stevens identified four properties for describing the scales of
measurement:

\begin{itemize}
\tightlist
\item
  \textbf{Identity}: each value has a unique meaning.
\item
  \textbf{Magnitude}: the values of the variable have an ordered
  relationship to one another, so there is a specific order to the
  variables.
\item
  \textbf{Equal intervals}: the data points along the scale are equally
  spaced, so the difference between data points one and two, is the same
  as data points three and four.
\item
  \textbf{A minimum value of zero}: the scale has a true zero point.
\end{itemize}

As previously said, Stevens identified four scales of measurement, that
is how variables are defined and categorised:

\begin{itemize}
\item
  \textbf{Nominal scale of measurement}: This scale has certain
  characteristics, but doesn't have any form of numerical meaning. The
  data can be placed into categories but can't be multiplied, divided,
  added or subtracted from one another. It's also not possible to
  measure the difference between data points. It defines only the
  identity property of data.\\
  Examples: Gender, Etnicity, Eye colour\ldots{}
\item
  \textbf{Ordinal scale of measurement}: It defines data that is placed
  in a specific order. While each value is ranked, there's no
  information that specifies what differentiates the categories from
  each other. These values can't be added to or subtracted from.\\
  Examples: satisfaction data points in a survey, where `one = happy,
  two = neutral and three = unhappy.'
\item
  \textbf{Interval scale of measurement}: The interval scale contains
  properties of nominal and ordered data, but the difference between
  data points can be quantified. This type of data shows both the order
  of the variables and the exact differences between the variables. They
  can be added to or subtracted from each other, but not multiplied or
  divided (For example, 40 degrees is not 20 degrees multiplied by
  two.).\\
  In this scale of measurement the zero is just a convention and not
  absolute, it is an existing value of the variable itself.
\item
  \textbf{Ratio scale of measurement}: This scale include properties
  from all four scales of measurement. The data is nominal and defined
  by an identity, can be classified in order, contains intervals and can
  be broken down into exact value. Weight, height and distance are all
  examples of ratio variables. Data in the ratio scale can be added,
  subtracted, divided and multiplied. Ratio scales also differ from
  interval scales in that the scale has a `true zero'. The number zero
  means that the data has no value point.\\
  An example of this is height or weight, as someone cannot be zero
  centimetres tall or weigh zero kilos.
\end{itemize}

\hypertarget{scales-and-questionnaires-development}{%
\subsection{Scales and Questionnaires
development}\label{scales-and-questionnaires-development}}

Measurement plays a vital role across scientific disciplines, with each
field creating specialized methods and tools tailored to its unique
subjects of study. In the behavioral and social sciences, the area
devoted to measurement is called psychometrics. This subfield
concentrates on evaluating psychological and social constructs, which
are most often assessed using questionnaires. Theaching how to build
effective questionnaires would require a specific course, but this is
out of the scope of this course. The following are some practical
guidelines that researchers can use to develop measurement scales and
questionnaires.

\hypertarget{determine-clearly-what-you-want-to-measure}{%
\subsubsection*{Determine Clearly What You Want to
Measure}\label{determine-clearly-what-you-want-to-measure}}
\addcontentsline{toc}{subsubsection}{Determine Clearly What You Want to
Measure}

Researchers often discover their initial ideas about what they want to
measure are vague, which can lead to costly changes later. Key questions
include whether the scale should be theory-based or explore new
directions, its level of specificity, and which aspects of the
phenomenon to emphasize.

\begin{itemize}
\item
  \textbf{Define the theory}: Basing scale development on relevant
  substantive theories is essential for clearly defining the construct
  being measured, particularly when dealing with abstract or
  non-observable phenomena. A theoretical basis helps establish the
  construct's boundaries, reducing the risk of the scale extending into
  unrelated areas. In the absence of an existing theory, developers
  should create a conceptual framework of their own---beginning with a
  precise definition and linking the new construct to related,
  established ones.
\item
  \textbf{Determine the level of specificity}: In psychometric scale
  development, it's important to consider how general or specific the
  measurement should be. This decision affects how well the scale works
  in predicting or relating to other variables. For example, if you're
  interested in general attitudes about personal control, a broad scale
  scale works well. But if you're studying beliefs about controlling a
  specific health issue, a focused scale is more appropriate.
\item
  \textbf{Define which aspects are enphasised}: Scale developers must
  clearly distinguish the target construct from related ones. Scales can
  be broad (e.g., general anxiety) or narrow (e.g., test anxiety).
  Including items outside the intended focus can lead to confusion or
  inaccurate measurement. For example, in health contexts, physical
  symptoms caused by an illness might be mistaken for psychological
  symptoms (like depression), leading to misleading results. Therefore,
  item selection should match the specific research purpose and avoid
  overlap with unrelated constructs.
\end{itemize}

\hypertarget{generate-an-item-pool}{%
\subsubsection*{Generate an Item Pool}\label{generate-an-item-pool}}
\addcontentsline{toc}{subsubsection}{Generate an Item Pool}

When developing a psychometric scale, items should be \textbf{carefully
selected} or created to match the specific construct you aim to measure.
That means you need a clear idea of what the scale is supposed to do,
and every item on the scale should reflect that goal.

Imagine the construct (like anxiety, motivation, or trust) as something
hidden or latent, which can't be observed directly. The items on your
scale are the visible signs or behaviors that reflect this hidden thing.
So, each item acts like a small ``test'' of how much of that construct a
person has. If your items truly measure the construct, then someone with
a high level of the trait should tend to score higher on all of them.

When constructing the item pool, it is important to consider the
following aspects:

\begin{itemize}
\item
  \textbf{The latent construct} A good scale includes multiple items to
  improve reliability, but every single item must still be strongly
  connected to the latent construct. You should think broadly and
  creatively when writing items to make sure they cover all the
  different ways the construct can be expressed---but without straying
  into measuring something else.\\
  A construct is a single, unified idea (like ``attitudes toward
  punishing drug abusers'') that can be thought of as causing how
  someone responds to related items. A category, on the other hand, is
  just a grouping of different constructs (like ``attitudes'' in
  general, or ``barriers to compliance'').\\
  Just because several items relate to the same category doesn't mean
  they measure the same underlying construct. For instance, ``Barriers
  to compliance'' is a category that can include many distinct things
  (fear of symptoms, cost concerns, distance to treatment, etc.). Each
  of these could be a separate construct with its own latent variable,
  so a scale that mixes these up wouldn't truly be unidimensional (i.e.,
  measuring just one thing).
\item
  \textbf{Redundancy} is crucial for reliability: multiple items allow
  common content to summate while idiosyncrasies cancel out. However,
  avoid superficial redundancy (e.g., minor wording changes, identical
  grammatical structures) which can inflate reliability estimates.
  Useful redundancy involves expressing the same core idea differently.
  Overly specific or redundant items within a broader scale can create
  subclusters (e.g., multiple specific anxiety items in a general
  emotion scale), potentially undermining unidimensionality and biasing
  the scale. This is less of a problem if the items match the scale's
  intended specificity.
\item
  \textbf{The number of items} Start with more items than planned for
  the final scale (e.g., 3-4 times as many) to allow for careful
  selection and ensure good internal consistency. An initial pool 50\%
  larger might suffice if items are hard to generate or fewer are needed
  for reliability. If the pool is too large, eliminate items based on
  criteria like lack of clarity or relevance.
\item
  \textbf{The wording} Including both positively worded items
  (indicating the presence of the construct) and negatively worded items
  (indicating its absence or low levels) is a common strategy to reduce
  acquiescence bias---the tendency of respondents to agree with
  statements regardless of their content. However, reversing the wording
  can sometimes confuse participants, particularly in general population
  or community samples, and this confusion may reduce the scale's
  reliability.
\end{itemize}

\begin{tcolorbox}[enhanced jigsaw, arc=.35mm, coltitle=black, left=2mm, toprule=.15mm, colback=white, colframe=quarto-callout-caution-color-frame, rightrule=.15mm, colbacktitle=quarto-callout-caution-color!10!white, opacityback=0, toptitle=1mm, bottomrule=.15mm, bottomtitle=1mm, breakable, titlerule=0mm, opacitybacktitle=0.6, title=\textcolor{quarto-callout-caution-color}{\faFire}\hspace{0.5em}{Caution}, leftrule=.75mm]

Reversing the wording of items (also known as reversed polarity) can
confuse respondents, especially if the items are complex or abstract, ot
if the respondents have lower reading comprehension or aren't used to
taking surveys.\\
This confusion can lead to inconsistent or inaccurate responses, which
lowers the reliability of the scale (i.e., how consistently it measures
the construct).

\end{tcolorbox}

\hypertarget{determine-the-format-for-measurement}{%
\subsubsection*{Determine the Format for
Measurement}\label{determine-the-format-for-measurement}}
\addcontentsline{toc}{subsubsection}{Determine the Format for
Measurement}

Defining the measurement format is a critical step in designing data
collection instruments like questionnaires and scales. This decision,
ideally made concurrently with item generation, impacts data quality,
variability, instrument sensitivity, and ultimately, research
conclusions.

Most scale items consist of two parts: a stem and a series of response
options. A kew aspect of the scale items is the number of response
options. A desiderable quality of a measurement scale is variability.
One way to increase opportunities for variability is to include lots of
scale items. Another way is to provide numerous respose options within
each item, especially with fewer items.

In this view, continuous formats (e.g., thermometer scales) offer many
gradations, and so increase the opportunities for variability. However,
too many options can exceed respondents' ability to meaningfully
discriminate, leading to ``false precision'' and increased error
variance. Researchers must balance the need for variability with
respondents' cognitive limitations.

Another issue the investigator has to concern with, is whether the
number of options should be even or odd. This choice depends on the type
of question. the type of response option, and the objectives of the
investigator.

An odd number of categories usually allows to express neutrality, while
an even number of categories forces a choice from the respondent. The
choice depends on whether allowing neutrality is desirable or should be
avoided.

There exist several ways to present items that are commonly used:

\begin{itemize}
\item
  \textbf{Likert Scale}: It is one of the most common item formats. The
  Liker Scale presents declarative statements with response options
  indicating degree of agreement (e.g., strongly disagree to strongly
  agree), and it is useful for measuring opinions, beliefs, attitudes.\\
  The statements should generally be moderately strong, since it is
  better to lead the respondent to not give responses near the center of
  the scale from the ``average'' respondent to maximize variance and
  discrimination.\\
  Options should represent roughly equal intervals. Likert Scales with
  5, 6, 7 categories are commonly used.
\item
  \textbf{Semantic Differential Scale}: It is used in reference ot one
  or more stimuli (e.g.~a group of people) followed by pairs of opposite
  adjectives (e.g., honest/dishonest) separated by several response
  lines/spaces. Respondents mark the space reflecting their evaluation.
  The adjectives can be bipolar (friendly/hostile) or unipolar
  (friendly/not friendly). To measure an underlying variable, multiple
  related adjective pairs can be used (e.g., honesty).
\item
  \textbf{Visual Analog Scale} (VAS): Presents a continuous line between
  two descriptors and the respondents mark a point on the line.
  Therefore, it is clear that this scale allows continuous scoring but
  it has to be noted that interpretation can be subjective, and
  comparisons across individuals may be difficult. An advantage of this
  type of scale is that it is shghly sensitive, so it useful for
  detecting subtle changes within individuals over time; moreover they
  may reduce reduce bias from recalling previous discrete responses.
\end{itemize}

\begin{itemize}
\tightlist
\item
  \textbf{Binary Options}: Offer two choices (e.g., agree/disagree,
  yes/no, check if applies). This type of option is simple for
  respondents but yields to minimal variability per item, therefore more
  items are required for obtaining an adequate scale variance. However,
  the ease of response may allow for more items to be administered.
\end{itemize}

\hypertarget{experts-review}{%
\subsubsection*{Experts' review}\label{experts-review}}
\addcontentsline{toc}{subsubsection}{Experts' review}

Expert review plays a key role in strengthening content validity during
scale development. By drawing on their subject-matter expertise,
reviewers help ensure that the items meaningfully represent the
construct.

Experts are typically asked to assess how well each item reflects the
construct definition, providing feedback that can confirm or refine the
conceptual framework. They also evaluate the clarity and precision of
item wording, offering suggestions to reduce ambiguity. In addition,
experts may highlight important aspects of the construct that have been
overlooked.

However, it's important to note that content experts may not be familiar
with psychometric principles. For instance, they might recommend
eliminating seemingly redundant items, not realizing that some
redundancy is intentional and necessary for reliability. While expert
input is highly valuable, final decisions should rest with the scale
developer, who must balance expert judgment with methodological rigor.

\hypertarget{subsequent-steps-in-scales-development}{%
\subsubsection*{Subsequent Steps in Scales
Development}\label{subsequent-steps-in-scales-development}}
\addcontentsline{toc}{subsubsection}{Subsequent Steps in Scales
Development}

Following the initial design of the questionnaire, including the
selection and construction of appropriate scales, the next crucial phase
involves preparing for validation and data collection. This includes
strategically incorporating additional items aimed at facilitating later
validation efforts, such as those designed to detect response biases or
to assess the questionnaire's construct validity by measuring
theoretically related concepts.

Subsequently, the questionnaire is administered to a development sample.
It's essential that this sample is sufficiently large and representative
of the target population to ensure stable results and minimize concerns
about subject variance.

Once the data is collected, a thorough evaluation of the individual
items is undertaken. This involves examining their intercorrelations to
ensure they are measuring a common underlying construct, addressing any
negatively correlated items through techniques like reverse scoring, and
assessing the correlation of each item with the overall scale.
Furthermore, the variance and means of the items are analyzed to ensure
they discriminate effectively among respondents. Factor analysis is
employed to confirm the dimensionality of the scale, and reliability,
often measured by Cronbach's alpha, is calculated to assess the internal
consistency of the items.

Finally, the length of the scale is optimized. This involves balancing
the need for brevity to reduce respondent burden with the desire for
higher reliability, which is generally associated with longer scales.
Weak items that negatively impact reliability are considered for
removal, and techniques like splitting the development sample for
cross-validation can be used to ensure the stability of the optimized
scale in new samples.

RANKINGS



\end{document}
